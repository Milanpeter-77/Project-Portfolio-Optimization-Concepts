%----------------------------------------------------------------------------------------
%	FONTS
%----------------------------------------------------------------------------------------

% Font size: 11pt
% Paragraph spacing: half
\documentclass[11pt, parskip=half]{scrartcl}

%----------------------------------------------------------------------------------------
%	PACKAGES AND OTHER DOCUMENT CONFIGURATIONS
%----------------------------------------------------------------------------------------

\usepackage{amsmath, amsfonts, amsthm} % Math packages
\usepackage{listings} % Code listings, with syntax highlighting
\usepackage{xcolor} % Color support for \textcolor
\usepackage[english]{babel} % English language hyphenation
\usepackage{graphicx} % Required for inserting images
\usepackage{booktabs} % Required for better horizontal rules in tables

\usepackage{float}   % for [H]
\usepackage{needspace} % to ensure enough room for heading+figure

\numberwithin{equation}{section} % Number equations within sections
\numberwithin{figure}{section}   % Number figures within sections
\numberwithin{table}{section}    % Number tables within sections

\setlength\parindent{0pt} % Removes all indentation from paragraphs

\usepackage{enumitem} % Required for list customisation
\setlist{noitemsep} % No spacing between list items

\usepackage{hyperref} % For hyperlinks in the PDF

%----------------------------------------------------------------------------------------
%	DOCUMENT MARGINS
%----------------------------------------------------------------------------------------

\usepackage{geometry} % Required for adjusting page dimensions and margins

\geometry{
	paper=a4paper,
	top=1.5cm,
	bottom=2cm,
	left=2cm,
	right=2cm,
	headheight=0.75cm,
	footskip=1.5cm,
	headsep=0.75cm,
	%showframe,
}

%----------------------------------------------------------------------------------------
%	SECTION TITLES
%----------------------------------------------------------------------------------------

% --- SECTION ---
% set I, II, III font styles

\setkomafont{section}{\normalfont\Large\bfseries}
%\setcounter{secnumdepth}{0}
\renewcommand\thesection{\Roman{section}} % Roman numerals for sections
%\renewcommand*\sectionformat{} % hide number only in the section heading
% --- SUBSECTION ---
\setkomafont{subsection}{\normalfont\bfseries}
% --- SUBSUBSECTION ---
\setkomafont{subsubsection}{\normalfont\itshape}
% --- PARAGRAPH ---
\setkomafont{paragraph}{\normalfont\scshape}

%----------------------------------------------------------------------------------------
%	HEADERS AND FOOTERS
%----------------------------------------------------------------------------------------

\usepackage{scrlayer-scrpage} % Required for customising headers and footers

\ohead*{} % Right header
\ihead*{} % Left header
\chead*{} % Centre header

\ofoot*{} % Right footer
\ifoot*{} % Left footer
\cfoot*{\pagemark} % Centre footer

%----------------------------------------------------------------------------------------
%	TITLE SECTION
%----------------------------------------------------------------------------------------

\title{	
	\normalfont\normalsize
	\vspace{-40pt}

	\rule{\linewidth}{0.5pt}\\ % Thin top horizontal rule
	\vspace{20pt}

	{\huge \textbf{Modern Portfolio Theory}}\\
	\vspace{10pt}
	{\LARGE \textit{A Theoretical Framework for Portfolio Optimization \\ and Risk Management}}\\

	\vspace{10pt}
	\rule{\linewidth}{1pt}\\ % Thick middle horizontal rule
	\vspace{0pt}

}

\author{
    \LARGE \textbf{Milan Peter} \\
    \large November 2025 \\
}

\date{}

\begin{document}

\maketitle % Print the title

\vspace{-3.5em}  % reduce vertical space between title and text

%----------------------------------------------------------------------------------------
%	INTRO
%----------------------------------------------------------------------------------------

\begin{quote}

Modern Portfolio Theory, born from Harry Markowitz's 1952 paper "Portfolio Selection", brought the scientific method into investing.
Its core message: is that risk and return must be analyzed together, not in isolation.
An investor should not simply choose assets with the highest expected returns, but rather combine assets so that their risks offset each other.
This is diversification: the only \textit{free lunch} in finance.
The theory constructs a quantitative relationship between expected return (reward), volatility (risk), and correlation (how assets move together).
The result is the efficient frontier, a mathematical boundary that separates rational portfolios (efficient) from inefficient ones.

\end{quote}

%----------------------------------------------------------------------------------------
%	SECTIONS
%----------------------------------------------------------------------------------------

\section{Basic Concepts}
\vspace{-1em}

%-------------------------------------------

\subsection{Expected Portfolio Return}
\vspace{-1em}

The starting point is expectation – what we think the portfolio will earn on average.
\[
E[R_p] = \mathbf{w}' \boldsymbol{\mu} = \sum_{i=1}^{N} w_i E[R_i]
\]
where
\begin{itemize}
	\item $E[R_p]$: expected portfolio return
	\item $E[R_i]$: expected return of asset $i$
	\item $\mathbf{w}'$: transpose of the weight vector
	\item $\boldsymbol{\mu}$: vector of expected asset returns
	\item $N$: number of assets in the portfolio
	\item $w_i$: portfolio weight of asset $i$, with $\sum_i w_i = 1$
\end{itemize}

The portfolio's expected return is the weighted average of the assets' expected returns. It grows linearly with the weights.

%-------------------------------------------

\subsection{Portfolio Volatility}
\vspace{-1em}

Risk is not additive. Two risky assets can make a safer portfolio if they move in opposite directions.
\[
\sigma_p = \sqrt{ \mathbf{w}' \boldsymbol{\Sigma} \mathbf{w} }
\]
where
\begin{itemize}
	\item $\sigma_p$: portfolio standard deviation (total risk)
	\item $\boldsymbol{\Sigma}$: covariance matrix, with elements $\sigma_{ij} = \text{Cov}(R_i, R_j)$
	\item $\mathbf{w}$: vector of portfolio weights
\end{itemize}

Portfolio risk depends on both individual asset volatilities and their correlations.
Even if every asset is volatile, their combined portfolio can be less volatile if they're not perfectly correlated.
Diversification works because the off-diagonal covariances can offset some of the total risk.
This is the mathematical essence of diversification.

%-------------------------------------------

\subsection{Sharpe Ratio}
\vspace{-1em}
Invented by William Sharpe (1966), this metric summarizes risk-adjusted performance:
\[
S = \frac{E[R_p] - R_f}{\sigma_p}
\]
where
\begin{itemize}
	\item $S$: Sharpe Ratio
	\item $E[R_p]$: expected portfolio return
	\item $R_f$: risk-free rate
	\item $\sigma_p$: portfolio volatility
\end{itemize}

Measures how much excess return per unit of risk the portfolio delivers.
It's the slope of the Capital Market Line (CML) – higher Sharpe means more efficient portfolios.

%-------------------------------------------

\section{Portfolio Optimization}
\vspace{-1em}

\subsection{Efficient Frontier}
\vspace{-1em}

Among all possible weight combinations, some portfolios are optimal: no other portfolio gives more expected return for the same volatility.
Formally:
\[
\min_{\mathbf{w}} \ \mathbf{w}'\boldsymbol{\Sigma}\mathbf{w}
\quad \text{s.t.} \quad
\mathbf{w}'\boldsymbol{\mu} = \mu_p,; \mathbf{w}'\mathbf{1}=1
\]
with
\begin{itemize}
	\item \textbf{objective}: minimize portfolio variance $\mathbf{w}'\boldsymbol{\Sigma}\mathbf{w}$
	\item \textbf{constraints}: achieve a target expected return $\mu_p$ and fully invest all wealth
\end{itemize}

Solving this optimization traces out the efficient frontier – the boundary of optimal portfolios offering the best return for each level of risk.
Portfolios below the curve are inefficient; those on the curve are the best possible trade-offs.

%-------------------------------------------

\subsection{Tangency Portfolio}
\vspace{-1em}

When a risk-free asset exists, we can draw a Capital Market Line (CML) from $R_f$ tangent to the frontier.
The tangency point defines the optimal risky portfolio:
\[
\mathbf{w}^* =
\frac{ \boldsymbol{\Sigma}^{-1}(\boldsymbol{\mu} - R_f \mathbf{1}) }
{ \mathbf{1}' \boldsymbol{\Sigma}^{-1}(\boldsymbol{\mu} - R_f \mathbf{1}) }
\]
where
\begin{itemize}
	\item $\mathbf{w}^*$: optimal weight vector of the tangency portfolio
	\item $R_f$: risk-free rate
	\item $\boldsymbol{\Sigma}^{-1}$: inverse covariance matrix
	\item $\boldsymbol{\mu}$: vector of expected returns
	\item $\mathbf{1}$: vector of ones
\end{itemize}

This portfolio lies at the tangency point between the risk-free line and the efficient frontier.
All rational investors hold this same risky portfolio but adjust their exposure to it by mixing in different amounts of the risk-free asset.

% -------------------
\subsection{Minimum Volatility Portfolio}
\vspace{-1em}

If we care only about minimizing risk (ignoring return expectations), we get:
\[
\mathbf{w}_{\min} =
\frac{ \boldsymbol{\Sigma}^{-1}\mathbf{1} }
{ \mathbf{1}' \boldsymbol{\Sigma}^{-1}\mathbf{1} }
\]
where
\begin{itemize}
	\item $\mathbf{w}_{\min}$: weight vector of the portfolio with the smallest variance
\end{itemize}

This is the global minimum variance portfolio – the most stable combination of assets.

%-------------------------------------------

\subsection{Daily and Annualized Return \& Volatility}
\vspace{-1em}

To compare returns and risk across time scales, we annualize:
\[
\mu_{\text{annual}} = \mu_{\text{daily}} \times 252,\quad
\sigma_{\text{annual}} = \sigma_{\text{daily}} \times \sqrt{252}
\]
where
\begin{itemize}
	\item $\mu_{\text{annual}}$: annualized mean return
	\item $\sigma_{\text{annual}}$: annualized standard deviation of returns
	\item 252: approximate number of trading days per year
\end{itemize}

It's essential when working with data of different frequencies.

% -------------------

\subsection{Visualizing Portfolio Optimization Concepts}
\vspace{-1em}

\begin{figure}[H]
	\centering
	\caption{Modern Portfolio Theory: Efficient Frontier and Capital Market Line}
	\includegraphics[width=\textwidth]{figures/efficient_frontier_cml.png}
	\label{fig:efficient-frontier-cml}
\end{figure}
\vspace{-1em}

This figure is a visual summary of Modern Portfolio Theory in action.
It translates the mathematics of portfolio optimization into a clear geometric picture.

On the horizontal axis is risk (annualized volatility, $\sigma_p$), and on the vertical axis expected return ($\mu_p$).
Each blue dot represents a single-asset portfolio – holding only one stock.
It can be seen that some assets, like TSLA, offer high potential return but also high volatility, while others, like MSFT or DIS, are steadier but yield less.

The blue curve is the efficient frontier.
It shows the best possible trade-off between risk and return achievable by combining these assets.
Any point below it is inefficient – you could earn more return for the same risk by moving up to the frontier.

The orange diamond marks the global minimum-variance portfolio – the least risky combination of assets.
The green star is the tangency portfolio: the point on the frontier that, when connected to the risk-free rate (here 2\%), gives the Capital Market Line (CML) shown in green.
The slope of that line is the Sharpe ratio, a measure of risk-adjusted performance.
Portfolios on this line combine the risk-free asset and the tangency portfolio in different proportions: investors with lower risk tolerance hold more of the risk-free asset; aggressive investors leverage beyond the star along the same line.

Together, this diagram illustrates how diversification allows investors to move from isolated risky points (individual stocks) to an efficient, optimized mix that maximizes expected return for any desired level of risk.

%-------------------------------------------

\section{Risk Measures}
\vspace{-1em}

%-------------------------------------------
\subsection{Value at Risk (VaR)}
\vspace{-1em}

Value at Risk is a statistical measure used to assess the risk of loss on an investment.
It estimates the maximum potential loss over a specified time period for a given confidence interval.
The formula for VaR is:
\[
\text{VaR}_{\alpha} = -(\mu_p + z_{\alpha}\sigma_p)
\]
where
\begin{itemize}
	\item $\alpha$: confidence level (usually 95\% or 99\%)
	\item $z_{\alpha}$: quantile of the standard normal distribution
	\item $\mu_p, \sigma_p$: portfolio mean and standard deviation
\end{itemize}

VaR can be calculated using historical simulation, variance-covariance, or Monte Carlo methods.


%-------------------------------------------

\subsection{Expected Shortfall (ES) / Conditional VaR}
\vspace{-1em}

VaR gives a threshold, while ES shows how bad the losses are beyond it:
\[
\text{ES}_{\alpha} = -\mu_p + \frac{\sigma_p \phi(z_{\alpha})}{1 - \alpha}
\]
where
\begin{itemize}
	\item $\phi(z_{\alpha})$: standard normal probability density at $z_{\alpha}$
\end{itemize}

Expected Shortfall averages the losses beyond the VaR threshold – a fuller measure of tail risk and downside vulnerability.

% -------------------
\subsection{Maximum Drawdown}
\vspace{-1em}

Measures the deepest decline from peak to trough:
\[
\text{MDD} =
\max_t \left( \frac{ \max_{s \le t} V_s - V_t }{ \max_{s \le t} V_s } \right)
\]
where
\begin{itemize}
	\item $\text{MDD}$: maximum drawdown
	\item $V_s$: portfolio value at time (s)
	\item $V_t$: portfolio value at time (t)
	\item $\max_{s \le t} V_s$: previous peak value up to time (t)
\end{itemize}

Captures the deepest cumulative loss.
Unlike volatility, it reflects path dependency – how losses unfold over time.

% -------------------
\subsection{Visualizing Risk Measures}
\vspace{-1em}

\begin{figure}[H]
	\centering
	\caption{Portfolio Daily VaR and Expected Shortfall (95\%)}
	\includegraphics[width=\textwidth]{figures/portfolio_var_es.png}
	\label{fig:portfolio-var-es}
\end{figure}
\vspace{-1em}

This figure visualizes Value at Risk (VaR) and Expected Shortfall (ES) – two central tools in financial risk management – applied to your daily portfolio returns.

The blue histogram shows the empirical distribution of daily returns, while the orange curve overlays a fitted normal distribution.
The shaded red area marks the left tail of losses beyond the 95\% confidence cutoff – the region where bad days live.

The solid red line is the historical $\text{VaR}_{\text{hist}}$, meaning that on 5\% of days, losses exceeded roughly 3.4\%.
The dashed orange line is the parametric $\text{VaR}_{\text{par}}$, estimated assuming returns are perfectly normal; it's slightly smaller because normal models understate extreme events.

The dotted red line and dash-dot orange line mark the corresponding Expected Shortfalls (ES) – the average losses beyond the VaR threshold.
ES captures tail severity, answering: "When things go wrong, how bad are they on average?"

Together, this plot contrasts the empirical vs. theoretical view of downside risk.
The left tail's heavier shape in the histogram relative to the normal curve reminds us that real-world returns are often not perfectly Gaussian – and that risk models relying solely on normal assumptions may underestimate the probability of extreme losses.

%----------------------------------------------------------------------------------------
%	REFERENCES
%----------------------------------------------------------------------------------------

\begin{thebibliography}{9}

\bibitem{jorion2007}
Jorion, P. (2007). \textit{Value at Risk: The New Benchmark for Managing Financial Risk}. 3rd edition. McGraw-Hill.

\bibitem{markowitz1952}
Markowitz, H. (1952). Portfolio Selection. \textit{The Journal of Finance} 7.1, pp. 77–91.

\bibitem{sharpe1966}
Sharpe, W. F. (1966). Mutual Fund Performance. \textit{The Journal of Business} 39.1, pp. 119–138.

\end{thebibliography}

%----------------------------------------------------------------------------------------
%	END OF DOCUMENT
%----------------------------------------------------------------------------------------

\end{document}
